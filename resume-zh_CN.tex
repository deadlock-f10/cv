% !TEX TS-program = xelatex
% !TEX encoding = UTF-8 Unicode
% !Mode:: "TeX:UTF-8"

\documentclass{resume}
\usepackage{zh_CN-Adobefonts_external} % Simplified Chinese Support using external fonts (./fonts/zh_CN-Adobe/)
%\usepackage{zh_CN-Adobefonts_internal} % Simplified Chinese Support using system fonts
\usepackage{linespacing_fix} % disable extra space before next section
\usepackage{cite}

\begin{document}
\pagenumbering{gobble} % suppress displaying page number

\name{樊一麟}

% {cs\_fyl@buaa.edu.cn}{13716159948}
% be careful of _ in emaill address
\contactInfo{cs\_fyl@buaa.edu.cn}{(+86) 137-161-59948}{}
% {E-mail}{mobilephone}
% keep the last empty braces!
%\contactInfo{xxx@yuanbin.me}{(+86) 131-221-87xxx}{}
 
\section{\faGraduationCap\  核心课程成绩}
\begin{itemize}
\item 核心理论课程\\
  操作系统 98,
  计算机组成原理 97,
  数据结构与算法 92,
  编译技术 92,
  计算机网络 92,
  数据库原理 91
  \item 实践类课程\\
计算机组成实验 100,
操作系统课程设计 97,
网络实验 94,
面向对象建模 94,
编译实验 90
\item 数学类课程\\
概率统计 100,
数学分析(一,二) 91,99,
离散数学(一,二,三) 88, 91, 92,
数学建模 90,
高等代数 89
\end{itemize}


\section{\faUsers\ 课外学习}
\datedsubsection{\textbf{MIT 6.828课程} }{2015年5月 -- 2015年7月}
\role{课程实践}{自学}
学习了MIT 6.828的操作系统理论知识,并在JOS---由MIT开发的基于X86的教学操作系统上---
\begin{itemize}
  \item 
	完成课程所有相关实验
  \item 改进完善了JOS中的简易shell
  \item 项目上传至 https://github.com/deadlock-f10/JOS
\end{itemize}

%\datedsubsection{\textbf{\LaTeX\ 简历模板}}{2014 年5月 -- 至今}
%\role{\LaTeX, Python}{个人项目}
\begin{onehalfspacing}
%优雅的 \LaTeX\ 简历模板, https://github.com/billryan/resume
%\begin{itemize}
%  \item 容易定制和扩展
%  \item 完善的 Unicode 字体支持,使用 \XeLaTeX\ 编译
%   \item 支持 FontAwesome 4.5.0
%\end{itemize}
%
%\datedsubsection{\textbf{自学}}{2016年7月}
%\textit{Graph isomorphism in quasipolynomial time}


\datedsubsection{\textbf{上海交大ACM班暑期课程}}{2016年7月}
主要完成了以下三门课程的学习:
\begin{itemize}
  \item 《概率与计算》 尹一通,南京大学
  \item 《近似算法》 贝小辉,新加坡南洋理工大学
  \item 《密码学》 石润婷,康奈尔大学
\end{itemize}

\datedsubsection{\textbf{数据结构复杂性下界论文研读}}{2016年7月 -- 至今}
\role{论文学习}{指导教授:尹一通,南京大学}
研读了数据结构复杂性的经典和前沿论文,其中包括静态和动态数据结构下界中的经典工具:
\begin{itemize}
  \item 丰富性引理(richness lemma)
  \item 回合消去引理(round elimination lemma)
  \item 数码(chronogram)技术
\end{itemize}
并撰写了关于这几项技术的综述报告

\end{onehalfspacing}

% Reference Test
%\datedsubsection{\textbf{Paper Title\cite{zaharia2012resilient}}}{May. 2015}
%An xxx optimized for xxx\cite{verma2015large}
%\begin{itemize}
%  \item main contribution
%\end{itemize}

\section{\faHeart\ 特别兴趣}
理论计算机:近似算法,随机算法,通讯复杂性,数据结构复杂性

\section{\faCogs\ IT 技能}
% increase linespacing [parsep=0.5ex]
\begin{itemize}[parsep=0.5ex]
  \item 编程语言: C > C++ == Java
  \item 平台: Linux
  \item 其他: \LaTeX, Mathmatica
\end{itemize}

\section{\faInfo\ 其他}
% increase linespacing [parsep=0.5ex]
\begin{itemize}[parsep=0.5ex]
  \item 语言: 英语 - 熟练(CET6 550(阅读满分))
\end{itemize}

%% Reference
%\newpage
%\bibliographystyle{IEEETran}
%\bibliography{mycite}
\end{document}
